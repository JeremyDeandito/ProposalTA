% ==========================================
% BAB II STUDI LITERATUR
% ==========================================
\chapter{STUDI LITERATUR}
\label{chap:studi-literatur}

\section{Sistem Informasi Akademik}
Sistem Informasi Akademik (SIAKAD) merupakan sistem informasi yang dirancang khusus untuk mengelola dan mengintegrasikan seluruh proses akademik di perguruan tinggi. SIAKAD berfungsi sebagai platform digital yang menghubungkan berbagai pemangku kepentingan (\textit{stakeholder}) akademik, yaitu mahasiswa, dosen, dan tenaga kependidikan, dalam menjalankan aktivitas akademik sehari-hari. Aktivitas tersebut meliputi pengisian Kartu Rencana Studi (KRS), pengelolaan nilai, pemantauan jadwal perkuliahan, pengumuman akademik, dan akses terhadap informasi akademik lainnya.

Dalam konteks pendidikan tinggi modern, SIAKAD telah berkembang menjadi komponen kunci dalam ekosistem digital kampus. Menurut \textcite{Sangsuwan2025}, sistem informasi digital dalam pendidikan menawarkan keunggulan signifikan dibandingkan sistem manual tradisional, terutama dalam hal aksesibilitas, pembaruan konten waktu nyata (\textit{real-time content updates}), pengalaman yang dipersonalisasi, dan keberlanjutan lingkungan. Studi terbaru menegaskan bahwa transformasi digital ini secara fundamental mengubah cara mahasiswa belajar, termasuk dampaknya terhadap hasil belajar, motivasi, kognisi, dan keterlibatan dalam lingkungan pendidikan \cite{Sangsuwan2025}. Oleh karena itu, SIAKAD yang dirancang dengan baik tidak hanya meningkatkan efisiensi administrasi akademik, tetapi juga berperan penting dalam meningkatkan pengalaman belajar mahasiswa dan efektivitas pengelolaan institusi.

\section{Desain Interaksi}
Desain Interaksi (\textit{Interaction Design}) adalah disiplin ilmu yang berfokus pada perancangan produk dan sistem interaktif yang mendukung cara orang berkomunikasi dan berinteraksi dalam kehidupan sehari-hari, baik untuk bekerja maupun untuk kegiatan lainnya \cite{rogers2023interaction}. Menurut \textcite{rogers2023interaction}, desain interaksi melibatkan pemahaman tentang siapa pengguna, apa yang mereka lakukan, dimana mereka melakukannya, dan bagaimana menciptakan pengalaman yang efektif, efisien, dan memuaskan ketika mereka berinteraksi dengan teknologi. Tujuan utama dari desain interaksi adalah menciptakan produk yang mudah dipelajari, efektif untuk digunakan, dan memberikan pengalaman pengguna yang menyenangkan.

Desain interaksi bersifat multidisipliner. Hal ini dikarenakan desain interaksi mengintegrasikan pengetahuan dari berbagai bidang termasuk desain produk, ilmu komputer, psikologi kognitif dan sosial, ergonomi, serta seni dan desain \cite{rogers2023interaction}. Pendekatan multidisipliner ini memungkinkan perancang untuk mempertimbangkan tidak hanya aspek teknis sistem, tetapi juga aspek kognitif, emosional, dan sosial dari pengalaman pengguna. Dalam pelaksanaannya, desain interaksi mencakup pengaturan alur kerja, struktur navigasi, jenis umpan balik, dan mekanisme penanganan kesalahan yang berkontribusi pada kualitas interaksi pengguna.

\subsection{Pendekatan Desain Interaksi}
Berdasarkan \textcite{rogers2023interaction} yang mengutip dari Dan Saffer, ada empat pendekatan utama dalam desain interaksi, yaitu \textit{User Centered Design}, \textit{Activity Centered Design}, \textit{Systems Design}, dan \textit{Genius Design}. Setiap pendekatan yang ada memiliki filosofi, fokus utama, serta metode yang berbeda-beda. Pemahaman terhadap keempat pendekatan ini penting untuk memilih strategi desain yang paling sesuai dengan konteks dan tujuan proyek.

\subsubsection{\textit{User Centered Design}}
\textit{User Centered Design} (UCD) adalah pendekatan desain yang menempatkan pengguna sebagai pusat dari seluruh proses perancangan. Produk yang dirancang dengan pendekatan ini harus disesuaikan dengan kebutuhan, karakteristik, dan konteks pengguna, bukan memaksa pengguna untuk menyesuaikan diri dengan sistem yang akan dirancang. UCD menekankan keterlibatan aktif pengguna dari tahapan awal hingga evaluasi akhir pengembangan produk \cite{rogers2023interaction}.

Menurut ISO 9241-210:2019, langkah-langkah yang dilakukan dalam \textit{User Centered Design} adalah sebagai berikut:

\begin{enumerate}
\item \textbf{\textit{Understanding and Specifying the Context of Use}}

Tahapan ini berfokus pada proses identifikasi siapa pengguna sistem, apa karakteristik mereka, tugas apa yang akan mereka kerjakan, serta kondisi lingkungan dimana sistem tersebut akan digunakan.

\item \textbf{\textit{Specifying the User Requirements}}

Tahapan ini berfokus pada perumusan kebutuhan fungsional dan nonfungsional yang harus dipenuhi agar sistem dapat membantu pengguna mencapai tujuannya dengan efektif. Proses perumusan harus berdasar pada konteks yang telah dipahami di tahapan sebelumnya.

\item \textbf{\textit{Producing Design Solutions}}

Pada tahapan ini, solusi solusi perancangan mulai dibuat, mulai dari konsep kasar, diagram alur, \textit{wireframe}, hingga purwarupa interaktif yang memvisualisasikan bagaimana sistem akan bekerja memenuhi kebutuhan pengguna.

\item \textbf{\textit{Evaluating the Design}}

Desain yang telah dibuat akan diuji dan dinilai secara langsung dengan melibatkan pengguna untuk memvalidasi apakah desain sudah memenuhi kebutuhan mereka atau tidak. Hasil evaluasi ini akan menentukan apakah desain sudah layak atau perlu direvisi kembali melalui proses iterasi yang sama.
\end{enumerate}

\subsubsection{\textit{Activity Centered Design}}
\textit{Activity Centered Design} (ACD) adalah pendekatan desain yang menjadikan aktivitas atau perilaku pengguna sebagai fokus utama, bukan individu pengguna itu sendiri. Pendekatan ini berasumsi bahwa dengan memahami struktur aktivitas lalu merancang sistem yang mendukung aktivitas tersebut secara optimal, pengalaman pengguna akan meningkat. ACD sangat berguna ketika aktivitas pengguna dalam sebuah sistem bersifat kompleks, melibatkan banyak peran berbeda, dan ketika efisiensi serta keandalan penyelesaian tugas menjadi prioritas utama \cite{rogers2023interaction}.

\subsubsection{\textit{Systems Design}}
\textit{Systems Design} adalah pendekatan desain yang mempertimbangkan sistem secara holistik, termasuk komponen teknis, sosial, dan organisasional yang saling berinteraksi. Pendekatan ini berfokus pada bagaimana berbagai elemen dalam sistem seperti manusia, teknologi, proses bisnis, dan lingkungan organisasi dapat bekerja bersama sebagai satu kesatuan. Pendekatan ini memastikan bahwa solusi desain tidak hanya memecahkan masalah lokal tetapi juga kompatibel dengan ekosistem organisasi yang lebih besar \cite{rogers2023interaction}.

\subsubsection{\textit{Genius Design}}
\textit{Genius Design} adalah pendekatan desain yang mengandalkan keahlian, intuisi, dan kreativitas individual desainer yang sangat berpengalaman untuk menciptakan solusi inovatif. Pendekatan ini sering dikaitkan dengan desainer "visioner" atau "genius" yang mampu menghasilkan desain terobosan tanpa melakukan riset pengguna yang ekstensif atau mengikuti proses desain yang terstruktur. Pengguna hanya memvalidasi ide-ide tersebut dan tidak dilibatkan selama proses desain itu sendiri \cite{rogers2023interaction}.

\section{\textit{User Interface}}
\textit{User Interface} (UI) adalah titik interaksi antara pengguna dan sistem digital, mencakup semua elemen visual, kontrol, dan komponen yang memungkinkan pengguna berkomunikasi dengan sistem. UI meliputi semua aspek yang dapat dilihat, didengar, atau disentuh oleh pengguna ketika berinteraksi dengan produk digital, termasuk \textit{layout} halaman, tipografi, warna, ikon, tombol, formulir input, menu navigasi, dan elemen visual lainnya.

Menurut \textcite{rogers2023interaction}, \textit{User Interface} yang baik harus memenuhi prinsip-prinsip berikut:

\begin{enumerate}
\item \textbf{Visibilitas}: Fungsi-fungsi penting harus terlihat jelas dan mudah ditemukan oleh pengguna.
\item \textbf{Konsistensi}: Elemen yang sama harus terlihat dan berfungsi dengan cara yang sama di seluruh aplikasi.
\item \textbf{\textit{Feedback}}: Sistem harus memberikan umpan balik yang jelas dan tepat waktu terhadap tindakan pengguna.
\item \textbf{\textit{Affordance}}: Desain harus mengomunikasikan dengan jelas bagaimana suatu elemen dapat digunakan.
\item \textbf{\textit{Constraints}}: Desain harus membatasi jenis interaksi yang mungkin dilakukan pada waktu tertentu untuk mencegah kesalahan.
\end{enumerate}

\section{\textit{User Experience}}
\textit{User Experience} (UX) adalah konsep yang lebih luas dari \textit{User Interface}. UX mencakup keseluruhan pengalaman seseorang ketika berinteraksi dengan suatu produk atau layanan \cite{rogers2023interaction}. Menurut standar ISO 9241-210:2019, \textit{User Experience} didefinisikan sebagai persepsi dan respons seseorang yang dihasilkan dari penggunaan dan/atau antisipasi penggunaan suatu produk, sistem, atau layanan.

UX tidak hanya terbatas pada aspek fungsional dan \textit{usability}, tetapi juga mencakup:

\begin{enumerate}
\item \textbf{Aspek emosional}: Perasaan pengguna saat dan setelah menggunakan produk.
\item \textbf{Aspek estetika}: Daya tarik visual dan keindahan desain.
\item \textbf{Aspek kualitas interaksi}: Tingkat kesenangan dan kepuasan dalam proses interaksi.
\item \textbf{Aspek temporal}: Perubahan pengalaman seiring waktu, mulai dari kesan pertama hingga penggunaan jangka panjang.
\item \textbf{Aspek kontekstual}: Pengaruh konteks penggunaan, termasuk lingkungan fisik, sosial, dan situasional.
\end{enumerate}

\section{\textit{User Experience Goals}}
\textit{User Experience Goals} adalah tujuan-tujuan yang ingin dicapai dalam menciptakan pengalaman pengguna yang berkualitas. \textit{User Experience Goals} lebih menekankan pada kualitas subjektif pengalaman pengguna \cite{rogers2023interaction}. Indikator \textit{User Experience Goals} yang diinginkan ditampilkan pada Tabel \ref{tbl:uxgoals-positive} dan Indikator \textit{User Experience Goals} yang tidak diinginkan ditampilkan pada Tabel \ref{tbl:uxgoals-negative}.



\begin{table}[h] % pilihan opsi yang disarankan: t = top, b = bottom, h = here
  \begin{tabular}{ | p{3cm} | p{3cm} | p{3cm} |}
	\hline
	\textit{Satisfying} 	& \textit{Helpful} 		& \textit{Fun} \\
	\hline
	\textit{Enjoyable} 	& \textit{Motivating} 		& \textit{Provocative} \\
  \hline
	\textit{Engaging} 	& \textit{Challenging} 		& \textit{Surprising} \\
  \hline
	\textit{Pleasureable} 	& \textit{Enhancing sociability} 		& \textit{Rewarding} \\
  \hline
	\textit{Exciting} 	& \textit{Supporting creativity} 		& \textit{Emotionally fulfilling} \\
  \hline
	\textit{Entertaining} 	& \textit{Cognitively stimulating} 		& \textit{Experiencing flow} \\
	\hline
	\end{tabular}
\caption{Tabel \textit{User Experience Goals} yang Diinginkan}
\label{tbl:uxgoals-positive}
\end{table}

\begin{table}[h] % pilihan opsi yang disarankan: t = top, b = bottom, h = here
  \begin{tabular}{ | p{3cm} | p{3cm} | p{3cm} |}
	\hline
	\textit{Boring} 	& \textit{Unpleasant} 		& \textit{Patronizing} \\
	\hline
	\textit{Frustrating} 	& \textit{Making one feel guilty} 		& \textit{Making one feel stupid} \\
  \hline
	\textit{Cutesy} 	& \textit{Gimmicky} 		& \textit{Childish} \\
  \hline
	\textit{Annoying} 	&  & \\
	\hline
	\end{tabular}
\caption{Tabel \textit{User Experience Goals} yang Tidak Diinginkan}
\label{tbl:uxgoals-negative}
\end{table}

\section{\textit{Usability Goals}}
\textit{Usability Goals} adalah kriteria objektif yang digunakan untuk mengukur seberapa baik sistem mendukung pengguna dalam menyelesaikan tugas-tugas mereka \cite{rogers2023interaction}. \textit{Usability Goals} lebih berfokus pada aspek fungsional dan kinerja sistem yang dapat diukur secara kuantitatif. Terdapat enam \textit{Usability Goals} utama, yaitu:

\begin{enumerate}
\item \textbf{\textit{Effectiveness}}: Kemampuan sistem mendukung pengguna dalam menyelesaikan tugas-tugas mereka dengan baik.
\item \textbf{\textit{Efficiency}}: Kecepatan dan usaha minimal yang dibutuhkan pengguna untuk menyelesaikan tugas.
\item \textbf{\textit{Safety}}: Kemampuan sistem melindungi pengguna dari kondisi berbahaya dan konsekuensi yang tidak diinginkan.
\item \textbf{\textit{Utility}}: Ketersediaan fungsi-fungsi yang tepat untuk memungkinkan pengguna melakukan hal yang mereka butuhkan atau inginkan.
\item \textbf{\textit{Learnability}}: Kemudahan bagi pengguna baru dalam mempelajari cara menggunakan sistem.
\item \textbf{\textit{Memorability}}: Kemudahan bagi pengguna yang sudah pernah menggunakan sistem untuk mengingat cara penggunaannya setelah lama tidak menggunakan sistem.
\end{enumerate}

\section{\textit{Usability Heuristics}}
\textit{Usability Heuristics} adalah prinsip-prinsip umum untuk desain interaksi yang dapat digunakan sebagai panduan dalam merancang antarmuka dan mengevaluasi \textit{usability} \cite{rogers2023interaction}. Berdasarkan \textcite{rogers2023interaction}, heuristik yang umum digunakan adalah 10 \textit{Usability Heuristics} yang dikembangkan oleh Jakob Nielsen. Berikut adalah 10 \textit{Usability Heuristics} Nielsen:

\begin{enumerate}
\item \textbf{\textit{Visibility of System Status}}: Sistem harus selalu memberi tahu pengguna tentang apa yang sedang terjadi melalui umpan balik yang tepat dalam waktu yang wajar.
\item \textbf{\textit{Match Between System and the Real World}}: Sistem harus berbicara dalam bahasa pengguna, menggunakan kata-kata, frasa, dan konsep yang familier bagi pengguna, bukan istilah teknis yang berorientasi sistem.
\item \textbf{\textit{User Control and Freedom}}: Pengguna memerlukan "pintu darurat" yang jelas untuk keluar dari keadaan yang tidak diinginkan akibat tindakan yang tidak disengaja, tanpa harus melalui dialog yang panjang.
\item \textbf{\textit{Consistency and Standards}}: Pengguna tidak harus bertanya-tanya tentang kesamaan makna dari kata, situasi, atau tindakan yang berbeda.
\item \textbf{\textit{Error Prevention}}: Mencegah masalah yang mungkin terjadi dengan implementasi desain yang tepat lebih baik daripada memberikan pesan saat kesalahan telah terjadi.
\item \textbf{\textit{Recognition Rather than Recall}}: Meminimalkan beban ingatan pengguna dengan membuat objek, tindakan, dan opsi terlihat supaya pengguna dapat mengenali pola desain yang dibuat.
\item \textbf{\textit{Flexibility and Efficiency of Use}}: Desain memberikan fleksibilitas dan efisiensi kepada pengguna untuk menyesuaikan tindakan yang sering dilakukan ketika menggunakan sistem.
\item \textbf{\textit{Aesthetic and Minimalist Design}}: Desain yang dibuat tidak mengandung informasi yang tidak relevan atau jarang dibutuhkan.
\item \textbf{\textit{Help Users Recognize, Diagnose, and Recover from Errors}}: Pesan kesalahan harus dinyatakan dalam bahasa yang jelas, tepat menunjukkan masalah, dan secara konstruktif menyarankan solusi.
\item \textbf{\textit{Help and Documentation}}: Desain sistem harus memiliki dokumentasi yang relevan untuk membantu pengguna dalam mempelajari segala hal yang terkait dengan sistem tersebut.
\end{enumerate}

\section{\textit{User Persona}}
\textit{User Persona} adalah representasi fiksi dari target pengguna yang dibuat berdasarkan data riset pengguna nyata \cite{rogers2023interaction}. \textit{User Persona} berfungsi sebagai alat komunikasi yang membantu tim desain untuk tetap fokus pada kebutuhan pengguna sepanjang proses desain dan pengembangan. \textit{User Persona} yang baik mencakup komponen berikut:

\begin{enumerate}
\item \textbf{Informasi Demografis}: Usia, pekerjaan, latar belakang pendidikan, lokasi geografis.
\item \textbf{Tujuan dan Motivasi}: Hal yang ingin dicapai pengguna dengan menggunakan sistem.
\item \textbf{Perilaku dan Preferensi}: Cara pengguna biasanya berinteraksi dengan teknologi, preferensi mereka, dan pola perilaku.
\item \textbf{Kebutuhan dan \textit{Pain Points}}: Masalah atau frustrasi yang dialami pengguna dalam mencapai tujuan mereka.
\item \textbf{Konteks Penggunaan}: Lingkungan dan situasi tempat pengguna akan menggunakan sistem.
\item \textbf{Kutipan atau Narasi}: Pernyataan yang mencerminkan perspektif dan sikap pengguna.
\item \textbf{Foto atau Ilustrasi}: Visual yang membantu membuat persona lebih terasa nyata dan mudah diingat.
\end{enumerate}

\section{\textit{Usability Testing}}
\textit{Usability Testing} adalah metode evaluasi dimana pengguna nyata mencoba menggunakan produk atau sistem untuk menyelesaikan tugas-tugas spesifik, sementara pengawas akan mengamati, mendengarkan, dan mencatat hasil interaksi tersebut \cite{rogers2023interaction}. Tujuan utama \textit{usability testing} adalah untuk mengidentifikasi masalah \textit{usability}, mengumpulkan data kuantitatif tentang kinerja pengguna, dan menentukan kepuasan pengguna terhadap produk. \textit{Usability Testing} dapat menggunakan berbagai metrik untuk mengukur \textit{usability}, termasuk metrik objektif dan metrik subjektif.

\subsection{\textit{System Usability Scale} (SUS)}
\textit{System Usability Scale} (SUS) adalah kuesioner standar untuk mengukur \textit{usability} yang dirasakan dari suatu sistem \cite{brooke1996sus}. SUS terdiri dari 10 pernyataan dengan skala Likert 5 poin, di mana responden menunjukkan tingkat persetujuan mereka terhadap setiap pernyataan. Adapun komposisi dari 10 pertanyaan tersebut adalah 5 pernyataan disusun secara positif dan 5 lainnya disusun secara negatif.

\begin{figure}[H]
	\centering
	\captionsetup{justification=centering}
	\includegraphics[width=0.9\textwidth]{image/ContohSUS.png}
	\caption{Contoh pertanyaan pada \textit{System Usability Scale}}
	\label{gambar:sus}
\end{figure}

Perhitungan skor SUS dilakukan melalui langkah-langkah berikut:

\begin{enumerate}
\item Untuk setiap pernyataan bernomor ganjil (1, 3, 5, 7, dan 9), skor jawaban pengguna dikurangi 1.
\item Untuk setiap pernyataan bernomor genap (2, 4, 6, 8, dan 10), nilai 5 dikurangi dengan skor jawaban pengguna.
\item Jumlahkan hasil perhitungan dari langkah 1 dan 2 untuk mendapatkan skor total sementara.
\item Kalikan skor total sementara dengan 2,5 untuk mendapatkan skor akhir SUS (rentang 0-100).
\end{enumerate}

Menurut \textcite{sauro2016quantifying}, skor di atas 80 dinilai sangat baik, skor di rentang 68-80 dinilai baik hingga cukup baik, skor di rentang 50-68 dinilai cukup buruk hingga buruk, dan skor dibawah 50 dinilai tidak dapat diterima.

\subsection{\textit{Single Ease Question} (SEQ)}
\textit{Single Ease Question} (SEQ) adalah instrumen pengujian \textit{usability} pasca-tugas yang digunakan untuk mengukur persepsi pengguna terhadap tingkat kesulitan tugas yang baru saja diselesaikan. Berbeda dengan SUS yang mengukur kegunaan sistem secara keseluruhan di akhir sesi, SEQ diberikan segera setelah pengguna menyelesaikan satu skenario tugas tertentu \cite{rogers2023interaction}.

Menurut \textcite{rogers2023interaction}, pengumpulan data dalam evaluasi tidak hanya dilakukan melalui kuesioner akhir, tetapi juga dapat dilakukan selama sesi berlangsung untuk menangkap respons pengguna terhadap fitur spesifik. SEQ umumnya menggunakan skala Likert 7 poin, mulai dari "Sangat Sulit" hingga "Sangat Mudah". Penggunaan metrik ini penting untuk mengidentifikasi bagian spesifik dari antarmuka yang menjadi \textit{pain points}, sehingga perbaikan dapat dilakukan secara lebih terarah pada fitur yang memiliki skor kemudahan rendah.

\begin{figure}[H]
	\centering
	\captionsetup{justification=centering}
	\includegraphics[width=0.9\textwidth]{image/ContohSEQ.png}
	\caption{Contoh \textit{Single Ease Question}}
	\label{gambar:seq}
\end{figure}

\subsection{\textit{Task Completion Rate} (TCR)}
\textit{Task Completion Rate} (TCR) adalah metrik utama untuk mengukur aspek efektivitas dalam tujuan \textit{usability}. Menurut \textcite{rogers2023interaction}, efektivitas merujuk pada seberapa baik sebuah sistem mendukung pengguna dalam mencapai tujuan mereka dengan akurat dan lengkap. TCR dihitung dengan persentase peserta yang berhasil menyelesaikan tugas yang diberikan tanpa bantuan kritis atau kesalahan fatal.

Rumus perhitungan TCR dapat dilihat pada Persamaan \ref{eq:tcr}.

\begin{equation}
\text{TCR} = \left( \frac{\text{Jumlah Tugas Berhasil}}{\text{Total Tugas yang Diberikan}} \right) \times 100\%
\label{eq:tcr}
\end{equation}

Nilai TCR memberikan gambaran objektif mengenai apakah desain antarmuka yang dirancang, seperti penempatan tombol atau alur navigasi, sudah cukup intuitif untuk memungkinkan pengguna menyelesaikan tujuannya secara mandiri.

\subsection{\textit{Time on Task} (ToT)}
\textit{Time on Task} (ToT) adalah metrik yang digunakan untuk mengukur aspek efisiensi dari sebuah sistem. Dalam konteks desain interaksi, \textcite{rogers2023interaction} menjelaskan efisiensi sebagai cara sistem mendukung pengguna dalam menyelesaikan tugas dengan sumber daya yang minimal, di mana waktu adalah salah satu sumber daya utamanya.

ToT mengukur durasi waktu yang dibutuhkan pengguna untuk menyelesaikan suatu tugas dari awal hingga akhir. Pengukuran ini dimulai saat pengguna mulai berinteraksi untuk mengerjakan tugas dan berhenti saat tugas berhasil diselesaikan. Waktu penyelesaian yang lebih singkat umumnya mengindikasikan antarmuka yang lebih efisien dan navigasi yang lebih jelas, asalkan penyelesaian tugas tersebut tetap akurat. Dalam evaluasi perancangan ulang (\textit{redesign}), penurunan rata-rata ToT dibandingkan dengan sistem lama seringkali menjadi indikator keberhasilan perbaikan antarmuka \cite{KhusnanHafidz2024}.

\section{Penelitian Terkait}
Penelitian pertama adalah penelitian oleh \textcite{KhusnanHafidz2024} yang meneliti perancangan ulang \textit{e-learning} di MTsN 16 Jakarta. Masalah utama yang diidentifikasi adalah kesulitan siswa menemukan informasi (\textit{findability}), seperti formulir tugas dan tenggat waktu, yang menyebabkan skor SUS awal sangat rendah (42,9).
Penelitian ini menerapkan metode UCD dengan teknik \textit{Hierarchical Task Analysis} (HTA) untuk memetakan ulang tugas kompleks menjadi langkah-langkah yang lebih sederhana. Solusi desain yang ditawarkan mencakup fitur \textit{Task Timeline} di dasbor utama dan restrukturisasi forum diskusi. Hasil evaluasi terhadap 61 responden menunjukkan peningkatan signifikan skor SUS menjadi 80,82 (kategori \textit{Good}), membuktikan bahwa perbaikan navigasi dan kejelasan informasi sangat krusial dalam sistem pembelajaran.

Penelitian kedua adalah penelitian oleh \textcite{AzZahra2024} yang mengevaluasi \textit{website} APBD Kota Bandung menggunakan metode \textit{Human Centered Design} (HCD). Masalah awal meliputi rendahnya kesadaran pengguna, antarmuka yang monoton, dan inefisiensi interaksi.
Penelitian ini menggunakan pendekatan etnografi untuk memahami konteks sosial pengguna. Solusi desain berfokus pada minimalisme dan penggunaan infografis untuk menyajikan data anggaran yang kompleks. Hasilnya, efisiensi interaksi meningkat dan kepuasan pengguna melonjak drastis, dengan skor SUS akhir mencapai 104 (kategori \textit{Excellent}). Peningkatan ini mengindikasikan bahwa penyederhanaan visual sangat efektif untuk transparansi data publik.

Penelitan ketiga adalah penelitian oleh \textcite{Ridho2023} yang melakukan perancangan ulang \textit{website} laboratorium HUMIC di Telkom University. Fokus utama penelitian ini adalah memperbaiki responsivitas \textit{mobile}, tata letak yang berantakan, dan navigasi yang membingungkan yang menyebabkan skor SUS awal hanya 43.
Metode UCD diterapkan dengan analisis \textit{Mental Model} untuk menyelaraskan navigasi dengan ekspektasi pengguna. Desain solusi difokuskan pada pendekatan \textit{Mobile-First}. Evaluasi menggunakan kombinasi \textit{Single Ease Question} (SEQ) dan SUS menunjukkan peningkatan skor SEQ dari 5,5 menjadi 6,4 dan skor SUS menjadi 80,5. Studi ini menegaskan bahwa responsivitas \textit{mobile} adalah faktor kunci dalam kredibilitas sistem akademik.

% \subsection{Gambar}
% Contoh gambar dapat dilihat pada Gambar \ref{gambar:jaringan}. Gambar dan judulnya diposisikan di tengah. Nomor gambar tidak diakhiri tanda titik. Gambar tersebut dibuat menggunakan aplikasi draw.io dan disimpan ke format PNG setelah dengan zoom setting pada angka 300\%. Ukuran gambar yang ditampilkan dapat diatur dengan mengubah nilai \textit{width} dalam sintaks \textit{includegraphics}.

% \begin{figure}[t] % pilihan opsi yang disarankan: t = top, b = bottom, h = here
% 	\centering
%   \captionsetup{justification=centering}
%     	\includegraphics[width=0.7\textwidth]{image/gambar1.png}
% 	\caption{Contoh gambar jaringan}
% 	\label{gambar:jaringan}
% \end{figure}

% Gambar umumnya tidak jelas atau kabur jika gambar tersebut:
% \begin{enumerate}[a.]
%   \item diperoleh dari hasil cropping pada suatu halaman buku atau situs web;
%   \item hasil pembesaran gambar yang gambar aslinya sebenarnya berukuran kecil; atau
%   \item disimpan dalam resolusi kecil
% \end{enumerate}
% Ketidakjelasan gambar ini dapat dilihat pada garis-garis diagram yang tidak tegas dan tulisan-tulisan dalam gambar yang tampak kabur dan kurang jelas terbaca.

% Untuk mendapatkan gambar yang tidak kabur (\textit{blur}), langkah-langkah berikut dapat digunakan:
% \begin{enumerate}[(a)]
% \item Gambar yang didapat di suatu pustaka atau referensi sebaiknya digambar ulang, misalnya menggunakan PowerPoint, Canva, Figma, draw.io, atau yang lainnya.
% \item Jika diagram atau ilustrasi digambar menggunakan draw.io, saat gambar disimpan ke format PNG atau JPG (\textit{export as}), lakukan \textit{zoom} ke minimal 300\% (\textit{the default value is} 100\%). 
% \item Jika diagram digambar dengan menggunakan PowerPoint, gambar dapat langsung di-\textit{copy-paste} ke Word.
% \end{enumerate}

% \subsection{Kajian Ilmiah Informatika dan Komputer \textit{Redesign E-learning with User Centered Design Method for Improved Accessibility Students}}
% dasasdas

% \subsection{\textit{Evaluation and Improvement of User Interface Design of Bandung City APBD Website Using Human Centered Design Method}}
% dsadadsa

% \subsection{\textit{Redesigning the User Interface of a University Laboratory Website Using the User-Centered Design Approach}}